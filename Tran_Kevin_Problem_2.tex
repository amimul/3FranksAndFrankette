\documentclass[letterpaper,12pt]{article}

\begin{document}

\paragraph{Suppose $x+\frac{1}{x}$ is an integer. Prove by induction that $x^n + \frac{1}{x^n}$ for every $n\geq1$ would also be an integer. }
\subparagraph{}
\paragraph{Before we start, let us note that we assume that $x$ is a valid value in order for instances of $x + \frac{1}{x}$ to yield a defined integer. We have also chosen to re-represent the problem using the recurrence relation that we found: $a_{n+1} = a_{n}*a_{1}-a_{n-1}$  with $a_{0}=2$ and $a_{1}=x+\frac{1}{x}$}. 
\subparagraph{}

\paragraph{For the first base case: If $n=1$, then $a_{1}=x^{n=1} + \frac{1}{x^{n=1}} = x + \frac{1}{x}$; an integer. }
\paragraph{For the second base case, let us consider the case when $n = 2$. If $n = 2$, then $a_{n=2}=a_{1}^2-a_{0}$, which is $(x+\frac{1}{x})^2-2$, expanded as: $x^2 + \frac{1}{x^2}+2-2$. Notice that $(x^2 + \frac{1}{x^2})$ is essentially $a_{1}^2$ so we can rewrite the expanded form further as $a_{1}^2 = a_{2}^2 + 2$ or $a_{2}^2 = a_{1}^2 - 2$. Since the first base case is an integer, an $integer^2 + 2$ or $integer * integer + 2$ will result an integer. }

\paragraph{We will now prove the hypothesis by induction, showing that $a_{n+1}$ (when $n=n+1$) is also an integer. }
\begin{itemize}
\item{$a_{n+1}=a_{n} * a_{1}$, so $a_{n+1}=(x^n + \frac{1}{x^n}) * (x^{n+1} + \frac{1}{x^{n+1}}) = \\\\(x^{n+1}+\frac{1}{x^{n+1}})+(x^{n-1}+\frac{1}{x^{n-1}})$}
\item{We can rewrite it as $a_{n} * a_{1} = a_{n+1} + a_{n-1}$ or $a_{n+1} = a_{n} * a_{1} - a_{n-1}$. }
\item{Testing for $a_{2}$, $a_{1+1} = a_{1}*a_{1}-a_{0}$. $a_{0}$ = $x^0 + \frac{1}{x^0}$ = 2 when the $x$ values result in an integer. This is equivalent to an $integer*integer-2$; which results in an integer. }
\item{Then for every increment of $a_{n}$, it will always calculate an $integer*integer-integer$, resulting in an integer. }
\end{itemize}
\paragraph{Thus, $x^n + \frac{1}{x^n}$ is an integer, for every $\textit{n}\geq1$. }

\end{document}