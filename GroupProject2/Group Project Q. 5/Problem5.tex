\documentclass[12pt]{article}\usepackage{amssymb,amsmath}\usepackage{graphicx}\usepackage{hyperref}\usepackage[margin=0.4in]{geometry}\newcommand{\ans}{\underline{\hspace{1cm}}}\newcommand{\Ans}{\hspace{.2in} \underline{\hspace{2cm}}}\renewcommand{\v}[1]{\vec{#1}}\newcommand{\m}[1]{\begin{bmatrix} #1 \end{bmatrix}}\newcounter{prob}\setcounter{prob}{1}\newcommand{\prob}{\arabic{prob}.\indent \addtocounter{prob}{1}}\usepackage{color}\hypersetup{colorlinks=true,linkcolor=blue,filecolor=magenta,      urlcolor=cyan,}\urlstyle{same}\begin{document}

a)$ T_n$ has $n(n-1)/2$ vertices.\\

$T_n$ has vertices created out of two element subsets with n elements where ordering doesn't matter. This can be represented by the choose function \( \binom{n}{2} \), which expanded is: \[\frac{n!}{(n-1)!\cdot2!}  =  \frac{n(n-1)(n-2)(n-3)...}{(n-2)(n-3)...1\cdot2}\]
Crossing out similar terms, we get: \[\frac{n(n-1)}{2}\]\\

b) Vertext of $T_n$ has degree $2n-4$\\

There are $n-1$ vertices containing the same element, as they are produced by pairing that element with the remaining elements in the set. As a vertex cannot connect with itself, it can pair with $n-2$ vertices containing that same element in the graph. As there are 2 elements in a vertex, there are $2\cdot(n-2)$ vertices that that vertex can pair with, or $2n-4$\\

c)If two vertices $x$ and $y$ are adjacent to each other in $T_n$, then there are $n-2$ vertices that are adjacent to both.\\

Two vertices, say $\big[a,b\big]$ and $\big[a,c\big]$ can connect with all vertices containing the similar element $a$, as they also connect with the vertex containing their opposing element $\big[b,c\big]$. As we've determined, there are $n-1$ vertices containing an element $a$, subtract the vertices we are looking at: $x$ and $y$, and adding the one vertex containing their opposing elements, there are $n-2$ vertices adjacent to both.\\

d) If two vertices $x$ and $y$ are $not$ adjacent to each other in $T_n$, then there are 4 vertices that are adjacent to both.\\

If there are two vertices $\big[a,b\big]$ and $\big[c,d\big]$, the only vertices adjacent to both are vertices formed for pairs of each other's variables: $\big[a,c\big]$, $\big[a,d\big]$, $\big[b,c\big]$, $\big[b,d\big]$. There are no other vertices both can connect to.\\

e) Are there any $T_n \big(n\geq3\big)$ for which $T_n$ is bipartite? Justify your answer.

No.\\

If $n = 3$, each vertex is adjacent to each other and would require 3 colors. On any graph larger than that we've proven above that any 2 adjacent vertices will have $n-2$ vertices in common, thus there will always be a need for 3 or greater number of colors.\\

f) Prove that $T_n$ is connected for any $n$. In particular, what is the longest distance between two vertices in $T_n$?\\

The longest distance between any two vertices is 2. As we've proven, for any two vertices not adjacent to each other, there are 4 adjacent to both. In this way there are only 2 edges max one needs to travel down to get to another vertex, otherwise the vertices are already adjacent.\\

h) Conjecture and prove: what is $ω\big(T_n\big)$\\

will be $n-1$. Every vertex with the same element as part of it's component will be able to connect with each other. As a single veriable can be in $n-1$ different pairs, there will be a clique of $n-1$. The only exception to this is $n=3$, as $\big[a,b\big]$, $\big[a,c\big]$, and the only other vertex connects to both of those vertices, $\big[b,c\big]$.\\

i)\includegraphics{graphs.png}\\

j) For what $n$ values is $|E\big(T_n\big)| > |E\big(\overline{T_n}\big)|$?\\

$\binom{n}{2}\cdot(2n-4)\>\binom{\binom{n}{2}}{2}-\binom{n}{2}(2n-4)$\\

$2\binom{n}{2}(2n-4)>\binom{\binom{n}{2}}{2}$	$k=\binom{n}{2}$\\

$2k(2n-4)>\binom{k}{2}$\\

$2k(2n-4)>\frac{k(k-1)}{2}$\\

$4k(2n-4)>k(k-1)$\\

$4(2n-4)>k-1$\\

$8n-15>k$\\

$8n-15>\frac{n(n-1)}{2}$\\

$16n-30>n^2-n$\\

$0>n^2-17n+30$\\

$0>(n-15)(n-2)$\\


When $n=16$

	$0>14$

When $n=14$

	$0>-12$

When $n=3$

	$0>-12$

When $n=1$

	$0>14$\\

For all $2< n < 15$,  $|E\big(T_n\big)| > |E\big(\overline{T_n}\big)|$\\

k) Is $T_n$ an induced subgraph of $T_{n+1}$?\\ $T_4 T_5$

Yes. As $n+1$ contains all of the elements in $n$ as well as 1 additional element, it will contain all the vertices in $T_n$, and as the adjacency rules remain the same, the same edges will form. As such, $T_n$ is a subgraph of $T_{n+1}$


\end{document}